\documentclass[a4paper, twocolumn, 11pt]{article}
\usepackage[utf8]{inputenc}
\usepackage[left=1.5cm, top=2.5cm, text={18cm, 25cm}]{geometry}
\usepackage[IL2]{fontenc}
\usepackage{times}
\usepackage{amsmath}
\usepackage{amsthm}
\usepackage{amssymb}
\usepackage{amstext}

\newtheorem{definice}{Definice}
\newtheorem{veta}{Věta}

\begin{document}

\begin{titlepage}

    \begin{center}
        \Huge
        \textsc{Fakulta informačních technologií\\ Vysoké učení technické v Brně}\\
        \vspace{\stretch{0,381966}}
        \LARGE{Typografie a publikování -- 2. projekt\\ Sazba dokumentů a matematických výrazů}
        \vspace{\stretch{0,61034}}
    \end{center}
    \LARGE{2021} 
    \hfill
    \LARGE{Pavel Heřmann (xherma34)}
\end{titlepage}

\section*{Úvod}
    V této úloze si vyzkoušíme sazbu titulní strany, matematických vzorců, prostředí a dalších
    textových struktur obvyklých pro technicky zaměřené texty (například rovnice (1)
    nebo Definice 1 na straně 1). Rovněž si vyzkoušíme používání odkazů \verb= \ref a \pageref.=
    
    \par
    
    Na titulní straně je využito sázení nadpisu podle optického středu s využitím zlatého řezu.
    Tento postup byl probírán na přednášce. Dále je použito odřádkování se zadanou relativní
    velikostí 0.4 em a 0.3 em.
    
    \par
  
    V případě, že budete potřebovat vyjádřit matematickou
    konstrukci nebo symbol a nebude se Vám dařit jej nalézt
    v samotném \LaTeX u, doporučuji prostudovat možnosti balíku maker \AmS - \LaTeX.
    
\section{Matematický text}

    Nejprve se podíváme na sázení matematických symbolů
    a výrazů v plynulém textu včetně sazby definic a vět s využitím balíku amsthm. Rovněž
    použijeme poznámku pod čarou s použitím příkazu \verb=\footnote=. Někdy je vhodné
    použít konstrukci \verb=\mbox{}=, která říká, že text nemá být
    zalomen.

        \begin{definice}
        Rozšířený zásobníkový automat (RZA) je definován jako sedmice tvaru A = ($Q, \Sigma, \Gamma, \delta, q_0, Z_0, F$), kde:
        \end{definice}
        
        \begin{itemize}
            \item $Q $ je konečná množina vnitřních (řídicích) stavů,
            \item $\Sigma $ je konečná vstupní abeceda,
            \item $\Gamma $ je konečná zásobníková abeceda,
            \item $\delta $ je přechodová funkce \(Q\times(\Sigma\cup\{\epsilon\})\times\Gamma^*\to
            2^{Q\times\Gamma^*}\),
            \item $q_0 \in Q  $ je počáteční stav, $Z_0 \in \Gamma $ je startovací symbol zásobníku a $F \subseteq Q $ je množina koncových stavů.
        \end{itemize}
        
    \par
    
    Nechť P = ($Q, \Sigma, \Gamma, \delta, q_0, Z_0, F$) je rozšířený zásobníkový automat. Konfigurací nazveme trojici ($q, w, \alpha $) $\in Q\times\Sigma^*\times\Gamma^* $, kde $q $ je aktuální stav vnitřního řízení, $w $ je dosud nezpracovaná část vstupního řetězce a $\alpha $
    $ = Z_{i1} Z_{i2} \dots Z_{ik} $ je obsah zásobníku\footnote{$Z_{i1}$ je vrchol zásobníku}.

    \subsection{Podsekce obsahující větu a odkaz}
        \begin{definice}
        Řetězec $w $ nad abecedou $\Sigma $ je přijat RZA A jestliže $(q_0, w, Z_0 )\overset{*}{\underset{A}{\vdash}} (q_F, \epsilon, \gamma) $ pro nějaké $\gamma\in\Gamma^* $
        a $q_F\in F $. Množinu L(A) = $\{ w\ |\ w $ je přijata RZA $ A\} $ 
        $\subseteq\Sigma^* $ nazýváme \textnormal{jazyk přijímaný RZA }A.
        \end{definice}
        
    \par
    \vspace{1em}
    Nyní si vyzkoušíme sazbu vět a důkazů opět s použitím balíku \verb=amsthm=.
    
    \begin{veta}
    Třída jazyků, které jsou přijímány ZA, odpovídá
    \textnormal{bezkontextovým jazykům.}
    \end{veta}
    
    \noindent
    Důkaz. V důkaze vyjdeme z Definice 1 a 2.

\section{Rovnice a odkazy}

    Složitější matematické formulace sázíme mimo plynulý
    text. Lze umístit několik výrazů na jeden řádek, ale pak je
    třeba tyto vhodně oddělit, například příkazem \verb=\quad=.
    
    \quad
    
    $\sqrt[i]{x_i^3} $\quad kde $x_i $ je $i $-té sudé číslo splňující\quad $x_i^{x_i^{i}^2+2} \leq 
    y_i^{x_i^4} $
    
    \quad
    
    V rovnici (\ref{1rovnice}) jsou využity tři typy závorek s různou explicitně definovanou velikostí.
    
    \begin{eqnarray}
        \label{1rovnice}
            \quad
            x&=& \bigg[\Big\{\big[a+b\big]*c\Big\}^d\oplus2\bigg]^{3/2} 
            \\
            y&=& \lim\limits_{x \to \infty} \frac{\frac{1}{log_{10^x}}}{\sin^2x+\cos^2x} 
            \nonumber
    \end{eqnarray}
    
    \par
    
    V této větě vidíme, jak vypadá implicitní vysázení limity $\lim_{x \to \infty}f(n) $ v
    normálním odstavci textu. Podobně je to i s dalšími symboly jako $\prod_{i=1}^n 2^i $ či $\bigcap_{A\in B}A $. V případě vzorců $\lim\limits_{n \to \infty}f(n) $ a $\prod\limits_{i=1}^{n} 2^i $ jsme si vynutili méně úspornou sazbu příkazem \verb=\limits.=
    
    \begin{eqnarray}
        \label{2rovnice}
            \int_{b}^{a}g(x)\ dx&=&- \int\limits_{a}^{b}f(x)\ \textnormal{d}x
    \end{eqnarray}
    
\section{Matice}
    Pro sázení matic se velmi často používá prostředí array a závorky (\verb=\left, \right=).
    
    $$
    \left(
        \begin{array}{ccc}
            a-b & \widehat{\xi+\omega} & \pi  \\
            \vec{\textbf{a}} & \overleftrightarrow{AC} & \hat\beta
        \end{array}
    \right)
    &=& 
        1 \Longleftrightarrow \mathcal{Q} = \mathbb{R}
    $$
    
    $$
    A&=& 
    \left|\left|
        \begin{array}{cccc}
            a_{11} & a_{12} & \dots & a_{1n} \\
            a_{21} & a_{22} & \dots & a_{2n} \\
            \vdots & \vdots & \ddots & \vdots \\
            a_{m1} & a_{m2} & \dots & a_{mn} 
        \end{array}
    \right|\right|
        &=&
    \left|
        \begin{array}{cc}
            t & u \\
            v & w
        \end{array}
    \right|
        &=&
    tw-uv
    $$
    
    Prostředí \verb=array= lze úspěšně využít i jinde.
    
    \par
    %DODĚLAT FUNKCI
    
    $$
    \binom{n}{k}
        &=&
    \Bigg\{
    \begin{array}{cl}
        0 & \quad \textnormal{pro} \; k < 0 \; \textnormal{nebo} \; k > n \\
        \frac{n!}{k!(n-k)!} & \quad \textnormal{pro} \; 0\leq k \leq n.
    \end{array}
    $$

\end{document}
